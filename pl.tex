\section{Some Middlebox Misconfiguration Errors}

A recent study in the network community indicates that 43\% of network failures occur due to misconfiguration errors~\cite{IMC:RJ13}.
For example, \figref{Miss1} shows a simple misconfiguration error when a new proxy is connected to the network, thereby
the proxy hides the source from the Firewall.
As a result host $A$ is no longer isolated from $B$.
\figref{LoadIDS} shows another bug which can happen when a new load-balancer is added in order to improve the network performance.
Lets assume that the IDS limits the number of paths from some host $A$.
If the load-balancer reroutes two packets from $A$ to different
As a result packets from the same host can be rerouted to different intrusion detection systems.
A remedy do this problem is to ensure that the load balancer uniquely route packets from the same source.
\begin{figure}
\caption{\label{fi:Miss1}%
A Proxy which leads to violation of network isolation.}
\end{figure}


\begin{figure}
\caption{\label{fi:LoadIDS}%
A load-balancer inserted before intrusion detection systems which leads to violation of network isolation.}
\end{figure}

% \section{Compositional Verification}
\section{Open Problems for Program Verification}
\section{What Does This Mean for Us?}
\label{sec:PL}

In this section, we argue that middleboxes are interesting from program language and verification perspectives.
In \secref{MidDefinition}, we define the semantics of middleboxes in a generic way.
The problem of verification of verifying a network of middleboxes is addressed in \secref{Verification}.
Finally, \secref{Comp} addresses the problem of compositional network verification.


\subsection{The Verification Problem}
\label{sec:Verification}
The area of program verification has gone a long way from the seminal works of Floyd, Hoare, and Dijkstra.
The biggest problem of program (and network) verification as already observed in \cite{popl:DeMilloLP77} is the absence of exact specification of what the system intends to do.
For example, we do not see how to exactly formalize the correctness of middleboxes like Intrusion Detected System(IDS).

\subsubsection{Network Isolation}
The most important safety problem for networks is \textbf{network isolation}, i.e., prove the absence of
paths between certain parts of the network.
Network isolation comes in many flavors.
The simplest form is requiring that certain hosts cannot send certain types of packets to certain hosts.
For example, different virtual networks should be disconnected even when they reside in the same physical network.
Also, certain hosts cannot send SSH packets.
A refinement of the isolation problem is assuring that the content of the packets sent from certain hosts cannot reach
certain hosts.
Such a violation can occur for example when proxies are inserted.

We believe that network isolation requirements can be easily specified by network administrators~\cite{Congress}.
Furthermore, they are similar to memory safety in the sense that violation of network isolation can actually be the consequence
of a logical error in the implementation.

\subsubsection{Misconfiguration Errors}


\subsubsection{The Complexity of Checking Network Correctness}
We now address the question of checking the correctness of network of middleboxes.
We begin with an obvious consequence of Rice Theorem:
\begin{theorem}
\begin{Name}Undecidability of network correctness\end{Name}
Consider a fixed topology of middleboxes.
Also assume that the semantics of every middlebox is expressed in Turing Complete programming language.
Then, it is undecidable to check any non-trivial property of networks, e.g., network isolation.
\end{theorem}

We are currently working on a programming language for specifying abstract middleboxes which allows to algorithmically check
interesting network properties including network isolation.
Despite this theorem, there may exist restricted programming languages for expressing the abstract effects of many mddleboxes that allows to algorithmically check many interesting
properties such as absence of loops and packet isolation.

\subsubsection{Correctness Preserving Transformations and Parametric Topologies}
It may be possible to show that certain correctness properties are preserved by the addition of certain middleboxes in certain ways.
For example, we may want show that the insertion of certain kinds of middleboxes e.g., WAN cannot lead to new violations of network isolation.
This will be very nice as it will give principles for changing network topologies.
Another open question is can one verify a parametric network of middleboxes?
For example, the Azure network seems to be built out many parallel paths for scalability.
Can we verify one path and then show that there are no interaction?
