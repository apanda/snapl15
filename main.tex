\documentclass[letterpaper,USenglish]{lipics}
\frenchspacing
%This is a template for producing LIPIcs articles.
%See lipics-manual.pdf for further information.
%for A4 paper format use option "a4paper", for US-letter use option "letterpaper"
%for british hyphenation rules use option "UKenglish", for american hyphenation rules use option "USenglish"
% for section-numbered lemmas etc., use "numberwithinsect"

\usepackage{microtype, outlines, xcolor, xspace, cite, units,amsmath}%if unwanted, comment out or use option "draft"
\usepackage{changebar}
% \newcommand{\lsyn}{\lbrack\!\lbrack}
\newcommand{\rsyn}{\rbrack\!\rbrack}
\newcommand{\semp}[1]{\lsyn #1 \rsyn}
%% references
\newcommand{\lnref}[1]{Line~\ref{ln:#1}}
\newcommand{\lemref}[1]{Lemma~\ref{lm:#1}}
\newcommand{\figref}[1]{Fig.~\ref{fi:#1}}
\newcommand{\defref}[1]{Definition~\ref{de:#1}}
\newcommand{\theref}[1]{Theorem~\ref{th:#1}}
\newcommand{\prref}[1]{Proposition~\ref{pr:#1}}
\newcommand{\tabref}[1]{Table~\ref{ta:#1}}
\newcommand{\secref}[1]{Section~\ref{sec:#1}}
\newcommand{\secrefs}[2]{Sections~\ref{sec:#1}--\ref{sec:#2}}
\newcommand{\ssecref}[1]{Sec.~\ref{sec:#1}}
\newcommand{\appref}[1]{Appendix~\ref{sec:#1}}
\newcommand{\exref}[1]{Example~\ref{ex:#1}}
\newcommand{\equref}[1]{eq~(\ref{eq:#1})}
\newcounter{my_counter}
\newcounter{tmy_counter}
\newcommand{\MyRoman}[1]%
        {\setcounter{my_counter}{#1}\roman{my_counter}}
\newcommand{\PMyRoman}[1]%
        {\rm (\MyRoman{#1})}
\newtheorem{MyTheorem}{Theorem}[section]
\newenvironment{The}{%
                \begin{MyTheorem}}{$\QED$\end{MyTheorem}}
\newenvironment{SThe}[1]%
{\par\noindent{\bf Theorem~\ref{The:#1}\/}\begin{em}}%
{\end{em}$\QED$}
\newtheorem{Proposition}[MyTheorem]{Proposition}
\newenvironment{Pro}{%
                \begin{Proposition}}{$\QED$\end{Proposition}}
\newenvironment{SPro}[1]%
{\par\noindent{\bf Proposition~\ref{Pro:#1}\/}\begin{em}}%
{\end{em}$\QED$}
\newtheorem{Lemma}[MyTheorem]{Lemma}
\newenvironment{Lem}{%
                \begin{Lemma}}{$\QED$\end{Lemma}}
\newenvironment{SLem}[1]%
{\par\noindent{\bf Lemma~\ref{Lem:#1}\/}\begin{em}}%
{\end{em}$\QED$}
\newenvironment{Claim}[1]%
{\par\noindent\begin{rm}{\em Claim\/}:~#1}%
{\end{rm}}
\newenvironment{CProof}[1]{\par\noindent %
\begin{rm}{\em Proof of Claim\/}#1:}{\end{rm}}
\newtheorem{Corollary}[MyTheorem]{Corollary}
\newenvironment{Cor}{\begin{Corollary}}{$\QED$\end{Corollary}}
\newtheorem{Conjecture}[MyTheorem]{Conjecture}
\newenvironment{Con}{\begin{Conjecture}}{$\QED$\end{Conjecture}}
\newtheorem{Observation}[MyTheorem]{Observation}
\newenvironment{Obs}{%
                \begin{Observation}}{$\QED$\end{Observation}}
\newtheorem{Definition}[MyTheorem]{Definition}
\newenvironment{Def}{\begin{Definition}}{$\QED$\end{Definition}}
\newtheorem{Example}[MyTheorem]{Example}
\newenvironment{Exa}{\begin{Example}\begin{rm}}%
                        {\end{rm}$\QED$\end{Example}}
\newenvironment{Ex}{\begin{quote}{\bf Example}.}%
                        {$\QED$\end{quote}}
\newenvironment{Note}{\par\noindent{\em Note\/}:}{}
\newenvironment{Remark}{\par\noindent{\bf Remark\/}.}{$\QED$}
\newenvironment{Proof}{\par\noindent %
{\em Proof:\/}\begin{rm}}{\end{rm}}
\newenvironment{ProofA}{%
\begin{rm}{\em Proof:\/}}{\end{rm}}
\newenvironment{ProofW}{\newline %
\begin{rm}{\em Proof:\/}}{\end{rm}}
\newenvironment{ProofIfA}{%
\begin{rm}{\em Proof of the if direction:\/}}{\end{rm}}
\newenvironment{ProofIf}{\newline %
\begin{rm}{\em Proof of the if direction:\/}}{\end{rm}}
\newenvironment{ProofOnlyIfA}{%
\begin{rm}{\em Proof of the only-if direction:\/}}{\end{rm}}
\newenvironment{ProofOnlyIf}{\newline %
\begin{rm}{\em Proof of the only-if direction:\/}}{\end{rm}}
\newenvironment{Sketch}{\par\noindent %
\begin{rm}{\em Sketch of Proof:\/}}{\end{rm}}
\newenvironment{Name}{\begin{bf}(}{)\end{bf}}
\newenvironment{Basis}{\newline {\em Basis\/}:}{}
\newenvironment{Hypothesis}{\newline {\em Induction hypothesis\/}:}{}
\newenvironment{Step}{\newline {\em Induction step\/}:}{}
\newenvironment{CSD}%


\newcommand{\eat}[1]{}
\newcommand{\allnotes}[1]{}
% To make the FIXMEs go away, comment out this line...
%\renewcommand{\allnotes}[1]{\textit{#1}}
\newcommand{\fixme}[1]{\allnotes{\bf\textcolor{red}{[#1]}}}
\newcommand{\notescott}[1]{\allnotes{\textcolor{blue}{[Scott: #1]}}}
\newcommand{\noteori}[1]{\allnotes{\textcolor{green}{[Ori: #1]}}}
\newcommand{\notemooly}[1]{\allnotes{\textcolor{purple}{[Mooly: #1]}}}
\newcommand{\notekaterina}[1]{\allnotes{\textcolor{gray}{[Katerina: #1]}}}
\newcommand{\notepanda}[1]{\allnotes{\textcolor{cyan}{[Panda: #1]}}}
\newcommand{\eg}{{\it e.g.,}\xspace}
\newcommand{\ie}{{\it i.e.,}\xspace}

%\graphicspath{{./graphics/}}%helpful if your graphic files are in another directory

\bibliographystyle{plain}% the recommended bibstyle

% Author macros::begin %%%%%%%%%%%%%%%%%%%%%%%%%%%%%%%%%%%%%%%%%%%%%%%%
\title{New Directions for Network Verification}
\author[1]{Aurojit Panda}
\author[2]{Katerina Argyraki}
\author[3]{Mooly Sagiv}
\author[4]{Michael Schapira}
\author[5]{Scott Shenker}
\affil[1]{UC Berkeley}
\affil[2]{EPFL}
\affil[3]{Tel Aviv University}
\affil[4]{Hebrew University of Jerusalem}
\affil[5]{UC Berkeley and ICSI}
% % \author{4]{Michael Schapira}
% \affil[1]{UC Berkeley}
% \affil[2]{TAU}
% \affil[3]{UC Berkeley, ICSI}
% \affil[4]{Hebrew University, Jerusalem}
% \author[1]{John Q. Open}
% \author[2]{Joan R. Access}
% \affil[1]{Dummy University Computing Laboratory\\
%   Address, Country\\
%   \texttt{open@dummyuni.org}}
% \affil[2]{Department of Informatics, Dummy College\\
%   Address, Country\\
%   \texttt{access@dummycollege.org}}
% \authorrunning{J.\,Q. Open and J.\,R. Access} %mandatory. First: Use abbreviated first/middle names. Second (only in severe cases): Use first author plus 'et. al.'
\authorrunning{A. Panda, K. Argyraki, M. Sagiv, M. Schapira, S. Shenker}
\Copyright{Authors}%mandatory, please use full first names. LIPIcs license is "CC-BY";  http://creativecommons.org/licenses/by/3.0/

\subjclass{C.2.6 Internetworking, F.3.1 Specifying and Verifying and Reasoning about Programs}
% mandatory: Please choose ACM 1998 classifications from http://www.acm.org/about/class/ccs98-html . E.g., cite as "F.1.1 Models of Computation".
\keywords{Middleboxes, Network Verification, Stateful Dataplane}% mandatory: Please provide 1-5 keywords
% Author macros::end %%%%%%%%%%%%%%%%%%%%%%%%%%%%%%%%%%%%%%%%%%%%%%%%%

%Editor-only macros:: begin (do not touch as author)%%%%%%%%%%%%%%%%%%%%%%%%%%%%%%%%%%
\serieslogo{}%please provide filename (without suffix)
\volumeinfo%(easychair interface)
  {Billy Editor and Bill Editors}% editors
  {2}% number of editors: 1, 2, ....
  {Conference title on which this volume is based on}% event
  {1}% volume
  {1}% issue
  {1}% starting page number
\EventShortName{}
% \DOI{10.4230/LIPIcs.xxx.yyy.p}% to be completed by the volume editor
% Editor-only macros::end %%%%%%%%%%%%%%%%%%%%%%%%%%%%%%%%%%%%%%%%%%%%%%%
\begin{document}
\setlength{\abovedisplayskip}{5pt}
\setlength{\belowdisplayskip}{5pt}
\setlength{\abovedisplayshortskip}{2pt}
\setlength{\belowdisplayshortskip}{2pt}

\maketitle
% \begin{abstract}
% \end{abstract}
% \begin{abstract}

% We define some future challenges for programming language techniques for verifying computer networks.
% We argue that some of the issues in computer networks such as efficient routing are well understood
% and require less attention.
% Also we list some functionality of computer networks such as intrusion detection which
% is not amenable to automatic program verification.
% In contrast, we identify essential network properties such as network isolation which are crucial for network functionality. Moreover, these properties can be violated by network miss-configuration errors.
% As a result, network administrators tends to configure networks in an overly-conservative way leading to
% huge resource consumption.
% Therefore, program verification can have a great impact on future networks by developing compile-time
% and run-time methods which will enforce these properties, thereby, leading to more effective networking.
%  \end{abstract}
% \section{Outline}
\begin{outline}
\1 Divide network into two halves
    \2 An arbitrarily complicated control plane; verifying properties for which can be undecidable.
    \2 A simpler dataplane, for which a variety of abstract models have been suggested.
\1 Lots of recent interesting in network verification; largely inspired by software-defined networking.
\1 Most of the commonly suggested models assume that the data plane in a network is stateless
    \2 FlowLog: Explicitly makes this assumption
    \2 NetKAT: The semantic allows combinations of stateless policies, the results of which are ultimately stateless. (Policies can only examine and depend on the first packet in a packet history) [NB: I am correct about this, I read the paper very carefully; however my wording is stupid]
    \2 Frenetic and HSA's network transfer functions: These are pure-functions from located packets to located packets, and hence don't allow state changes.
...
\1 In these models, the assumption, mistakenly adopted from early papers in SDN is that any stateful computation is carried out at the controller.
    \2 A variety of languages and verification tools verify the data plane configuration generated by these controllers. Tools from above help reasoning about these configurations.
    \2 More importantly, various extensions look at doing relatively fast, incremental checks on this data path configuration.
\1 But the reality in the network is a bit more complicated
    \2 Even very simple examples like Learning Switches are not implemented using the control plane
    \2 High latency, not high enough throughput makes controllers ill-suited to handle these examples.
    \2 More importantly, middleboxes abound:
        \3 Used to provide desirable security and performance optimization functions.
        \3 Can be arbitrarily complicated.
        \3 Importantly, stateful, behavior evolves over time.
    \2 Verification (and all the other tasks for which we are using these network models) is still an important concern in networks with these boxes.
\1 Gaps when we consider stateful networks
    \2 Unrepresentable in the current models, the models themselves need to be expanded.
    \2 Harder to prove decidability
    \2 Must reason about unbounded sequences of packets; long running mutable system.
\1 Challenge to the community
    \2 Come up with a set of assumptions and tools that help us reason about these networks.
\1 A first idea
    \2 Limit the total amount of state we have to reason about. A few steps to this.
    \2 Abstract middleboxes as loop free constructs. Loop freedom here means we only need to consider finite subsequences of packets, maybe getting us closer to decidability when analyzing a single middlebox. [This is actually risky, I have to work through this some more, but I think it is mostly correct] The number of sequences to be analyzed is itself a function of the description size.
        \3 Any middlebox is: simple abstract model which is generic and uses ``abstract'' data types + oracle (implementation of abstract-datatypes)
    \2 Consider loop-free toplogies (between a source-destination pair): this ensures that we only have to deal with finite combinations of middleboxes, each of which only needs 
    \2 RONO, to reason about a single, directed path through the network. This helps in a few ways:
        \3 Reduces the number of paths through a single middlebox we need to consider.
        \3 Eliminates natural loops: at most only need to consider a linear path to the destination and back (which can be considered a loop-free path).
    \2 Combine these three, we can reason about a composition of a finite set of middleboxes, each of whose behavior depends on a finite amount of state. Need to explore a finite set to decide if a policy is violated.
\1 Problems
    \2 Extend RONO naturally to tasks other than reachability.
\end{outline}
\section{Introduction}
\label{sec:introduction}

Verification --- by which we mean the general practice of checking the correctness of a computer-based system before it is put into use --- was first developed to
check the correctness of hardware, and is now increasingly used in the software development process. While networks have been around for many decades and are now an essential piece of our computational infrastructure, only recently has verification been applied to ensure their correctness.\footnote{There has been much work on verifying network protocols and their implementations, but until recently almost none on verifying a given network configuration.} As a result, there is now a growing literature on systems that can verify that the current or proposed network configuration (as represented by router forwarding tables) obey various important invariants (such as no routing loops or deadends).
These systems --- which allow network operators to verify that their networks will operate correctly, in terms of some well-defined invariants -- represent a valuable, and long overdue, step forward for networking, which for too long was satisfied with not only {\em best-effort} service but also {\em best-guess} configuration. In this position paper we critically review this recent progress, propose some next steps towards a more complete form of network verification, and end with a discussion of the formal questions posed to this community by the network verification agenda.

In the rest of this section, we provide some necessary background on networks and the current verification techniques. The application of verification to networking coincided with the rise of Software-Defined Networking (SDN). While not essential for the use of verification in networks, SDN provides a useful platform on which to deploy these tools so we discuss verification within the SDN context. Networks are comprised of two planes: the {\em data plane} which decides how packets are handled locally by each router (based on the local forwarding state and other information, such as state generated by previous packets), and the {\em control plane} which is a global process that computes and updates the local forwarding state in each router. In legacy networks, both planes are implemented in routers (with the data plane being the forwarding code or datapath, and the control plane being the global routing algorithm), but in SDN there is a clean separation between the two planes. The SDN control plane is logically centralized, and implemented in a few servers (called controllers) that compute and then install the necessary forwarding state. SDN-controlled routers only implement the data plane, executing a very simple datapath (OpenFlow \cite{openflow}) in which the routing state is a set of
$\langle \textit{match}, \textit{action} \rangle$ flow entries: all packets with headers matching the $match$ entry are subject to the specified $action$ which is often either to forward out a specific port (perhaps with a slightly modified header) or to drop the packet.
One can think of this network state as the {\em configuration} of a router.

The first wave of verification tools \cite{anteater,khurshid2012veriflow,oldhsa,kazemian2013real} analyzed the global behavior of a network made up of switches obeying this simple forwarding model. As a packet travels through the network, its next-hop is dictated by the routing state in the current router; thus, this network-wide behavior can be thought of as the composition of the routing state in each router. These early verification tools would take a snapshot of network state (either that which is already in the network, or that which the control is poised to insert into the network) and then verify whether some basic invariants held. These invariants (which are specified by the network operator) are typically quite simple and few in number: reachability (e.g., packets from host A can reach host B), isolation (e.g., packets from host A cannot reach host B), loop-freedom (no packet enters into an infinite loop), and no dead-ends (no packet arrives at a router which cannot forward it to another router or to the end-destination). Subsequent network verification tools (e.g., \cite{guha2013machine,anderson2014netkat,flowlog}) make the same assumptions about the datapath, but generalize along various other dimensions.

All of these tools leverage the fact that the simple forwarding model renders the datapath {\em immutable}; by this we mean that the forwarding behavior does not change until the control plane explicitly alters the routing state (which happens on relatively slow time scales). Thus, one can verify the invariants before each control-plane-initiated change and know that the network will always enforce the operator-specified invariants.

While the notion of an immutable datapath supported by an assemblage of routers makes verification tractable, it does not reflect reality. Modern enterprise networks are comprised of roughly $\nicefrac{2}{3}$ routers\footnote{In this paper we do not distinguish between routers and switches, since they obey similar forwarding models.} and $\nicefrac{1}{3}$ {\em middleboxes}~\cite{sherry2012making}.  Middleboxes -- such as Firewalls, WAN Optimizers, Transcoders, Proxies, Load Balancers, Intrusion Detection Systems (IDS) and the like --  are the most common way to insert new functionality in the network datapath, and are commonly used to improve network performance and security.\footnote{We should note that the NFV movement is moving middleboxes out of separate physical machines and into VMs that can be hosted on a cluster of servers; however, nothing in the move from physical to virtual middleboxes changes our story.} 

Just as the configuration of router is the state it uses to make forwarding decisions, the configuration of a middlebox is its set of policies (e.g., drop all Skype packets). The configuration of a network is the configurations of all of its routers, all of its middleboxes, and the topology of the physical network connecting these elements. The goal of verification is to ensure that a given network configuration supports a given invariant.

While useful, middleboxes are a common source of errors in the network~\cite{potharaju2013demystifying}, with middlebox misconfigurations account for over $40\%$ of all major
incidents in networks. Thus, one cannot ignore middleboxes when verifying network configurations.

However, middleboxes do not adhere to the simple forwarding model in routers. Many middleboxes have a {\em mutable} datapath, in which the handling of a packet depends not just on immutable forwarding state, but also on the sequence of previously encountered packets (e.g., a firewall packets from a flow into a network, but only if it has previously seen an outgoing packet from that flow).  This dependence on the packet histories renders the datapath quite mutable (changing on packet timescales). This prevents the use of current verification techniques, because the control over packet behaviors is no longer centralized in the control plane but is, in the most general case, dependent on the packet histories seen by every middlebox.


Thus, we must find a way to verify network behavior in the presence of middleboxes, which means finding verification techniques that can deal with mutable datapaths. Since the forwarding behavior of a mutable datapath can depend arbitrarily on the past packet history, middleboxes render the network Turing-complete. Thus, any verification technique that copes with middleboxes will look much more like general program verification than the current generation of network verification tools. There are two main technical challenges to building the next generation of network verification tools:
\begin{itemize}
\item How do we model these mutable datapaths so that the complexity of verification is tractable?
\item How can we feasibly analyze a network made up of these mutable datapaths?
\end{itemize}
We address these two challenges in the following sections, and then discuss  how to formalize this approach and end by describing a set of open questions.
\section{How to Model Middleboxes?}
\label{sec:mbmodel}
The natural approach for verifying mutable datapaths would be to apply standard program verification techniques to the code in each middlebox (and then extend this to the network as a whole, which is the problem we address in the next section). The practical problem with this approach is that middlebox code is typically proprietary, and any approach that relies on middlebox vendors releasing their code is doomed to fail.
Moreover, there is a deeper conceptual problem with this approach. The invariants specified by network operators often use abstractions, such as user identity, host identity, application-type (of the traffic), and whether or not the traffic is ``suspicious'' (\eg after deep packet inspection). In fact, recent efforts to build policy languages are built around a similar set of abstractions~\cite{congress}.

The correctness of these abstractions often {\em cannot} be fundamentally verified (\eg a middlebox in the middle of the network cannot always know for sure which host the packet came from given the various forms of spoofing or relaying available) or even precisely defined (\eg what is suspicious traffic?). Yet these abstractions are quite useful (and already widely used) in practice, and operators are willing to live with their approximations (\eg various techniques can be used to limit spoofing so that in some contexts host identification can rely on IP or MAC addresses without great risk).

\cbstart
To allow reasoning in terms of high level abstractions without worrying about the various approximations that go into their definition, we model middleboxes in two parts: a reasonably simple abstract model that captures the action of a middlebox in terms of high-level primitives and an {\em oracle} that is described by the set of abstractions it supports. The oracle maps packets to one or more abstract classes (\eg this packet is from a Skype flow from host A and user X to host B and user Y), while the abstract model describes how the middlebox forwards packets belonging to different abstract classes (\eg a middlebox might be configured to drop all suspicious packets, or only allow packets from host A to reach host B but no other hosts). For instance, for an IDS that identifies suspicious packets and forwards them to a scrubbing box, the oracle part of the model determines which packets are suspicious and the abstract model is what dictates that such packets are forwarded to a scrubbing box.
\cbend

The oracles in different middleboxes may use very different techniques to implement these abstractions. While operators care about the quality of this mapping, the goal of our network verification approach is to check that a network configuration correctly enforces invariants {\em assuming that the oracles are correct}. Also, different oracles may  support different sets of abstractions (\eg some firewalls may be able to identify Skype traffic, and others not), and this would be described as part of the middlebox model.

\cbstart
In contrast, the abstract models are fairly generic in the sense that the abstract model of a firewall applies to most firewalls. The degree of detail in these abstract models depends, to some degree, on the kind of invariants one wants to check.  The basic network invariants of reachability and isolation only require that the abstract model describe the forwarding behavior (\eg if and where each packet is forwarded). Our initial target is verifying these basic invariants, as these properties are by far the most important safety property provided by networks. If one wants to support performance-oriented invariants then the abstract model must include timing information (\eg what packet delays might occur), and other extensions are needed to consider invariants that address simultaneity (two properties always hold at the same time). For simplicity, we do not consider such extension here.
\cbend

Separating middlebox models into an abstract model and an oracle has several advantages.
\begin{itemize}
\item It captures the fact that there are a limited number of middlebox ``types'', with many implementations of each. The abstract model applies to all of these implementations (and is fairly simple in nature), and implementations mainly differ in the abstractions and features offered by the oracle.  Thus, our verification approach --- which asks whether invariants are enforced assuming the oracles are right --- can be applied independent of the implementations.
\item It differentiates between improvements in the oracle (\eg adding new abstractions to recognize application types), which is what consumes the bulk of the development effort, and verifying correctness of the network configuration.
\item It could change the vendor ecosystem by allowing (or requiring) vendors to provide the abstract model (and a description of which abstractions their oracle supports) along with their middlebox. Network operators could then perform verification, while vendors could keep their implementations private.
\item While it would be useful for vendors to {\em verify} that their code obeys the abstract model (using standard code verification methods), what makes this approach particularly appealing is that vendors and operators alike can {\em enforce} that the middleboxes obey the abstract model. The abstract models (and we have built several of them) are so simple that they execute much faster than the actual middlebox implementation, so one can run these abstract models in parallel and ensure that the middleboxes take no action that does not obey the abstract model.
\end{itemize}

\section{Network-Wide Verification}
\label{sec:modelnet}

Modeling individual middleboxes is only the first step; our ultimate goal is to verify network-wide invariants in a network containing middleboxes and routers.
There are numerous technical challenges (such as how to deal with loops), but in this short position paper we focus on the most challenging one: how can we feasibly perform verification in very large networks (containing on the order of thousands of middleboxes and routers)? To see why this is hard, consider the case of an isolation invariant (packets from host A cannot reach host B) in a network with many middleboxes. Since middlebox behavior depends not just on their configuration but on the packet history they've seen (since their datapath is mutable), verifying that this invariant holds, even if we ignore the possibility of packet loops,
involves checking that there is no sequence of packets --- involving packets sent from anywhere in the network at any time --- that includes a packet from host A reaching host B.


Without additional assumptions, this is not feasible as the forwarding of packets from host A to host B can arbitrarily depend on other hosts (\eg on whether some other host C previously sent packets to another host D). However, the most common classes of middleboxes have an important property: the handling of a packet from host A to host B depends only on the sequence of packets (seen by the middlebox) between hosts A and B. That is, many mutable datapaths exhibit a useful kind of locality.
For instance, in IP
firewalls, the middlebox tracks established connections, and allows packets to pass if they are either explicitly permitted by
policy or belong to an established connection. A connection between host A and B can only be established by host A and B. We therefore do not need to consider the
actions of any other hosts in the network. We call middleboxes whose behavior for a pair of hosts depends only on the traffic sent between these hosts ``RONO (Rest-Of-Network Oblivious) middleboxes''. See Section~\ref{sec:formal} for a formal definition of RONO. RONO is not a rare property; in fact, most middleboxes (including firewalls, WAN optimizers, load balancers, and others) are RONO.  Moreover, one can verify whether a middlebox is RONO by statically analyzing its abstract model.


Note that in many practical cases, 
the composition of RONO middleboxes is also RONO.  As a result, in a network containing only RONO
middleboxes we can verify reachability properties on a small subset of the network (the path between two hosts) and these properties would equivalently hold in
the context of the wider network. We have leveraged this fact to verify correctness of a network containing 30,000 middleboxes in under two minutes.

%\input{comp}
\section{Common Myths about Networks and SDN Verification}

We next highlight some common myths about SDN networks and contrast them with the reality of today's networks.

% \medskip{}

\subparagraph*{Myth \#1:} \emph{SDN networks only have controllers and OpenFlow routers, with all complicated (particularly mutable) packet processing done at the controller.} The early SDN literature~\cite{gude2008nox, monsanto2013composing} showed examples where anything expressible using OpenFlow rules was pushed down to routers, while anything more complex was implemented in the controller. Complicated functionality included both complex processing that could not be performed at routers and simple tasks that required mutability (\eg learning switches and stateful firewalls). However, doing this processing at the controller does not scale and, in reality, middleboxes are used to provide most of the processing functions not implementable on routers, and most routers provide some mutable behavior (e.g., learning switch).

\subparagraph*{Myth \#2:} \emph{Centralization, as provided by SDN, is what makes current network verification efforts possible.} Centralization is neither necessary nor sufficient for network verification. {\underline{Not necessary}:} Verification in a network with immutable datapaths only requires being able to access router forwarding state, and current commercial network verification efforts can do this in legacy networks by using commonly available commands to read this forwarding state. {\underline{Not sufficient}:} Regardless of SDN, current network verification efforts cannot verify networks that have middleboxes with mutable datapaths (which describes almost all real networks).


\subparagraph*{Myth \#3:} \emph{Middleboxes are an aberration that will be eliminated by the rise of SDN.} Quite the opposite is true. Not only are middleboxes here to stay, but SDN itself has been evolving to incorporate middleboxes~\cite{scottI2talk}. Furthermore, recently there has been an effort to move middleboxes from dedicated hardware (which is time-consuming to deploy) into virtual machines that can be deployed on quicker timescales, on existing hardware, and at lower cost. This effort is generally described as NFV (network function virtualization), and has gained significant traction commercially (comparable to or exceeding that of SDN), and recent efforts at defining a common configuration language, \eg Congress`\cite{congress}, treat middleboxes (virtualized or not) as first-class network citizens.


\subparagraph*{Myth \#4:} \emph{We should write all network code and configuration in declarative languages, because their use makes verification easy.} In general, reasoning about declarative languages is undecidable~\cite{Halevy}. It is true that verification is easy for declarative programs that do not use recursive rules (e.g., Congress~\cite{congress} or NLOG programs), even in the presence of mutable states. But then, verification is equally easy for imperative programs (e.g., Python, Java, or C programs) that honor certain restrictions, e.g., do not use loops. So, in the end, it is unclear that declarative languages can make a practical difference in verification. Some argue that declarative programs are easier to read and debug, once a programmer gets used to them. On the other hand, their readability becomes questionable in the presence of side-effects.

% \medskip

Once one discards these myths, it becomes clear that network-verification efforts must directly confront the presence of mutable datapaths. While the approach described here may not be optimal, it is currently the only one that confronts the reality of today's and tomorrow's networks. It is time to take the next step in network verification.

\section{Formalizing the Mutable Data Plane}
\label{sec:formal}
The previous sections argued why a new approach to network verification is needed and briefly outlined what it might look like.
In this section, we sketch a concrete way to formalize and prove interesting  properties of networks of middleboxes.

\begin{lstlisting}[caption={Model for an IDS},label=list:ids,captionpos=t,float,abovecaptionskip=-\medskipamount,
                    numbers=left,
                    morekeywords={oracle, model, when, default, state, forward},belowskip=-0.1in]
oracle suspicious? (packet: Packet) : Boolean;


model ids (p: Packet) = {
  when suspicious?(p) =>
      forward {}
  default =>
      forward {p}
}
\end{lstlisting}

% \subsection{Verification Problem}
DeMillo et al.~\cite{popl:DeMilloLP77} previously argued that specifying the desired behavior of a program (or network) is hard, and is a show-stopper for
verification. Indeed, the lack of a precise specification is a major problem for program and network verification. 

The primary function of networks is to allow hosts to communicate with each other. Reachabality, the property that a certain class of packets sent from host $A$ can reach host $B$, and its converse, isolation, are fundamental to networks: all useful networks must satisfy some set of reachability properties and their verification is thus universally important. In the rest of this section we limit our discussion to Reachability invariants.

We formally state these invariants using temporal logic where we assume fairness, \ie we assume any continuously enabled transition will eventually occur. 
First, we define two relations: $Send(n, p)$ indicating some network entity (node) $n$ sent
a packet $p$ (at some time), and $Recv(n, p)$ indicating node $n$ received packet $p$. Given these relations any reachability property can be expressed in
LTL (Linear Temporal Logic) as
\begin{align*}
\forall p\in \text{Packet}: \Box (Send(src, p) \land Predicate(p) \implies \Diamond(Recv(dest, p)))
\end{align*}
This temporal logic statement says that a packet $p$ sent by $src$ which satisfies $Predicate$ is eventually received by $dest$.
Similarly, isolation can be formally expressed by requiring that a packet sent by $src$, satisfying $Predicate$ is never received by $dest$:
\begin{align*}
\forall p\in \text{Packet}: \Box (Send(src, p) \land Predicate(p) \implies \Box(\neg Recv(dest, p)))
\end{align*}

$Predicate$ in the definitions above is specified using the same abstractions used to specify network policies, \ie either in terms of packet
header fields (source, destination, etc.) or in terms of the abstraction provided by a middlebox Oracle (\S\ref{sec:mbmodel}). For example, a property
saying no SSH traffic can reach a server $d$ can be expressed as
\begin{align}
\forall s\in \text{Node}: \Box (Send(s, p) \land ssh(p) \implies \Box(\neg Recv(d, p))) \label{eq:isossh}
\end{align}

where $ssh(p)$ returns true if an Oracle classifies the packet as belonging to an SSH connection.
We reason about reachability and isolation properties assuming that the Oracles are correct.
Verifying the isolation property in Equation~(\ref{eq:isossh}) therefore requires answering the question:
``assuming SSH traffic is correctly identified, can a packet belonging to an SSH connection reach $d$?''

% \subsection{Formalizing Middlebox Semantics}

% \notepanda{I think we should cut this bit, or move it elsewhere}
As stated earlier, we model a middlebox as an oracle and a simple abstract model. The Oracle provides abstractions that are used to specify the
properties being checked. We expect models for middleboxes (which include both an oracle and a generic model) to be specified
using a constrained programming language. Listing~\ref{list:ids} shows an example of such a specification
for an intrusion detecion system (IDS). The IDS oracle provides one abstraction, \texttt{suspicious?}, defined in the first line. The abstract model is
defined in Lines 4 -- 9 and uses this abstraction. First we check to see if the packet is suspicious (the Oracle's decision here might be based on what
packets it has seen previously), and  drop the packet (Line 6) or forward the packet (Line 9) depending on the value returned by the Oracle.
% \notepanda{End of where we would cut until}

Next, we develop abstract semantics for middleboxes. We use these to reason about general properties (such as composition) that apply to all middleboxes. Our semantic model is defined over a potentially infinite set of packets, $P$.  We augment packets to include information about their location (\ie the
middlebox or
switch port). We also define the operator $\doteq$, such that $p_1 \doteq p_2$ implies that packets $p_1$ and $p_2$ are identical except for their location.
Finally, we use $P^*$ to represent the set of all (potentially unbounded) sequence of packets.
The abstract model middlebox $m$, is then a function $m\colon P\times P^* \to 2^P$ which takes a packet ($p\in P$) and a history ($h\in P^*$) 
of all the packets that have
previously been processed by $m$ and produces a (possibly empty) set of packets $m(p, h)$. Given this model a switch is a simple function for which
$m(p, h) \doteq \left\{p\right\}$. Similarly a simple firewall, $f$ (whose decision process is represented by $allowed$) can be expressed as
\begin{align*}
 f(p, h) = \begin{cases}
    \left\{p'\right\}\ p' \doteq p & \mbox{if } allowed(p, h)\\
    \left\{\right\} & otherwise
 \end{cases}
\end{align*}

% \subsection{Composing Middleboxes}
Ideally, we would like to be able to reason about the network \emph{compositionally}, \ie the correctness of the network should follow from
the correctness of smaller, simpler components. Compositional reasoning can reduce the cost of verification and enable incremental verification
of changes in the network. Compositionality has been important for making verification tractable in other domains, for instance the use of
rely-guarantees~\cite{tse:MisraC81,ifip:Jones83}, was important for enabling verification of concurrent programs.

We start by defining what it means to be able to compositionally verify a network. Let us
define the union of two networks $N_1$ and $N_2$ in the natural way, \ie $N_1\cup N_2$ contains the union of all nodesand links in each of $N_1$ and $N_2$.
Consider two networks: $N_1$, where property $P_1$ holds (represented as $N_1\models P_1$); and $N_2$, where property $P_2$ holds. We can compose the proofs
for properties $P_1$ and $P_2$ if and only if  both $P_1$ and $P_2$ hold for $N_1 \cup N_2$. More formally, we can verify properties $P_1$ and $P_2$
compositionally for network $N$ if for any $N_1 \subset N$ and $N_2\subset N$
\begin{align*}
N_1 \models P_1,  N_2\models P_2\\
\hline
(N_1 \cup N_2) \models P_1\land P_2
\end{align*}

\begin{figure}
\centering
\includegraphics[width=0.9\textwidth]{figures/rono_example.pdf}
\caption{Example where networks are not composable with respect to reachability properties.}
\label{fig:compose_fail}
\vspace{-0.15in}
\end{figure}

Generally, one cannot perform compositional verification of reachability properties. For instance,
consider the example in Figure~\ref{fig:compose_fail}. The cache in this example records all requests to and the corresponding responses from $S$.
On receiving a new request,  the cache checks to see if it has previously recorded a response for this request, in which case it returns the saved response;
otherwise the cached forwards the request, unmodified, to the firewall.
 The firewall drops all requests sent from $A$ to $S$, but otherwise forwards all other requests and responses unmodified.
In network $N_1$, $A$ can never receive a response from $S$ (thus is isolated). However in the composed network $N_1\cup N_2$, if $B$ sends a request $r$ and
receives response $r'$ from $S$, then $A$ can also request $r$ and receive $r'$.

One key insight is that despite being impossible in general, there exists an important subset of networks where compositional reasoning can be used
to verify reachability properties. These networks contain only Rest-Of
Network Oblivious middleboxes (\S\ref{sec:modelnet}), middleboxes whose behavior for a pair of hosts depends only on the traffic sent between these hosts.
More formally, we define the restriction $h|_{(A, B)}$ of a packet history $h\in P^*$ to be the subsequence of $h$ containing only those packets that were
sent between host $A$ and $B$. We define a middlebox $m$ to be RONO if and only if
\begin{align*}
f(p, h) = f(p, h|_{(A, B)}) &\ \  \forall p: p.src = A \land p.dest = B \text{   and}\\
f(p, h) = f(p, h|_{(A, B)}) &\ \  \forall p: p.src = B \land p.dest = A
\end{align*}

Not all compositions of RONO middleboxes are themselves RONO. In~\cite{corr:PandaLASS14}, we have identified conditions where a network of RONO middleboxes
supports compositional verification.

\section{Open Problems for Network Verification}
Finally, we present some open problems that we have encountered while looking at how to verify mutable dataplanes. This list is not 
exhaustive, but is rather an attempt to list the first set of hurdles that need to be crossed given this new network verification agenda.

\textbf{Decidability of Verification} 
When processing a packet, a middlebox might access potentially unbounded state. This prevents the use
of finite-state model checking, and other verification techniques are undecidable for general programs in this class. We are
currently working on a limited programming language that is rich enough to specify all exiting middleboxes, while simultaneously
enabling verification of some interesting network properties, including reachability properties.

% Our abstract models for middleboxes such Firewall an IDSs have infinite state.
% Therefore, in general it may be undecidable to check properties like network isolation.
% We are currently working on a programming language for specifying abstract middleboxes, which allows to algorithmically check
% interesting network properties, including network isolation.


\textbf{Specification}
Specifying verification properties and middleboxes remains an open question in general. While we have provided some tools
that allow us to specify and check reachability properties, extending this to other invariants, for example performance based
invariants remains challenging. How middleboxes and properties are specified also has a huge impact on verification time and
decidability. Therefore, it is crucial to pick specifications that are rich enough to permit operators to express interesting and
useful properties, yet narrow enough to permit automated reasoning.

\textbf{Correctness Preserving Transformations and Parametric Toplogies}
It might be possible to extend some of our results on compositional reasoning to show that the addition of certain types of middleboxes
can never affect some class of invariants. We know this is true for some middleboxes in reality, the addition of a stateless firewall
can never affect an isolation invariant (though it might invalidate some reachability invariants). Developing a theory for when this 
holds might be useful in developing techniques to help simplify network changes. 

Compositional verification similarly hints at the fact that we might be able to design some ``parametric'' topologies, where one can provably
show that a change to one network path cannot possibly affect another. It is easy to see that this is easily achieved for stateless datapaths.
However, the complexity of what can built using parametric toplogies, when mutable data paths are used remains an open question.
%\section{Some Middlebox Misconfiguration Errors}

A recent study in the network community indicates that 43\% of network failures occur due to misconfiguration errors~\cite{IMC:RJ13}.
For example, \figref{Miss1} shows a simple misconfiguration error when a new proxy is connected to the network, thereby
the proxy hides the source from the Firewall.
As a result host $A$ is no longer isolated from $B$.
\figref{LoadIDS} shows another bug which can happen when a new load-balancer is added in order to improve the network performance.
Lets assume that the IDS limits the number of paths from some host $A$.
If the load-balancer reroutes two packets from $A$ to different
As a result packets from the same host can be rerouted to different intrusion detection systems.
A remedy do this problem is to ensure that the load balancer uniquely route packets from the same source.
\begin{figure}
\caption{\label{fi:Miss1}%
A Proxy which leads to violation of network isolation.}
\end{figure}


\begin{figure}
\caption{\label{fi:LoadIDS}%
A load-balancer inserted before intrusion detection systems which leads to violation of network isolation.}
\end{figure}

% \section{Compositional Verification}
\section{Open Problems for Program Verification}
\section{What Does This Mean for Us?}
\label{sec:PL}

In this section, we argue that middleboxes are interesting from program language and verification perspectives.
In \secref{MidDefinition}, we define the semantics of middleboxes in a generic way.
The problem of verification of verifying a network of middleboxes is addressed in \secref{Verification}.
Finally, \secref{Comp} addresses the problem of compositional network verification.


\subsection{The Verification Problem}
\label{sec:Verification}
The area of program verification has gone a long way from the seminal works of Floyd, Hoare, and Dijkstra.
The biggest problem of program (and network) verification as already observed in \cite{popl:DeMilloLP77} is the absence of exact specification of what the system intends to do.
For example, we do not see how to exactly formalize the correctness of middleboxes like Intrusion Detected System(IDS).

\subsubsection{Network Isolation}
The most important safety problem for networks is \textbf{network isolation}, i.e., prove the absence of
paths between certain parts of the network.
Network isolation comes in many flavors.
The simplest form is requiring that certain hosts cannot send certain types of packets to certain hosts.
For example, different virtual networks should be disconnected even when they reside in the same physical network.
Also, certain hosts cannot send SSH packets.
A refinement of the isolation problem is assuring that the content of the packets sent from certain hosts cannot reach
certain hosts.
Such a violation can occur for example when proxies are inserted.

We believe that network isolation requirements can be easily specified by network administrators~\cite{Congress}.
Furthermore, they are similar to memory safety in the sense that violation of network isolation can actually be the consequence
of a logical error in the implementation.

\subsubsection{Misconfiguration Errors}


\subsubsection{The Complexity of Checking Network Correctness}
We now address the question of checking the correctness of network of middleboxes.
We begin with an obvious consequence of Rice Theorem:
\begin{theorem}
\begin{Name}Undecidability of network correctness\end{Name}
Consider a fixed topology of middleboxes.
Also assume that the semantics of every middlebox is expressed in Turing Complete programming language.
Then, it is undecidable to check any non-trivial property of networks, e.g., network isolation.
\end{theorem}

We are currently working on a programming language for specifying abstract middleboxes which allows to algorithmically check
interesting network properties including network isolation.
Despite this theorem, there may exist restricted programming languages for expressing the abstract effects of many mddleboxes that allows to algorithmically check many interesting
properties such as absence of loops and packet isolation.

\subsubsection{Correctness Preserving Transformations and Parametric Topologies}
It may be possible to show that certain correctness properties are preserved by the addition of certain middleboxes in certain ways.
For example, we may want show that the insertion of certain kinds of middleboxes e.g., WAN cannot lead to new violations of network isolation.
This will be very nice as it will give principles for changing network topologies.
Another open question is can one verify a parametric network of middleboxes?
For example, the Azure network seems to be built out many parallel paths for scalability.
Can we verify one path and then show that there are no interaction?

\bibliography{snapl}
% \newpage
% \appendix
% \section{Implications of State}
\label{sec:stateful}
\eat{
\begin{outline}
\1 Previous approaches bad
    \2 Models cannot easily account for state
    \2 Hard to use fuunctional and declarative stuff for stateful networks
\1 Must resort to some form of model checking
    \2 Not feasible
        \3 Large code base
        \3 Many middleboxes
\end{outline}}


\eat{\fixme{SDN inspired model}\notepanda{kind of done, eliminate once fixed}
Verifying computer networks has generated a lot of interest in the  verification community, recent work such as Header-Space
Analysis (HSA)~\cite{kazemian2013real}, Veriflow~\cite{}, Frenetic~\cite{guha2013machine},  NetKAT~\cite{anderson2014netkat}, FlowLog~\cite{flowlog} have
presented techniques and tools for verifying the correctness of networks. These efforts have been informed by the early SDN literature, and assume that the
control-plane
is written using Turing complete languages executed on traditional computers, while the data plane is composed of a collection of relatively simple stateless
switches. }
\fixme{Not sure if this should be here, or presented at all, but in case we wanted to clarify the switch model}
Any processing that cannot be performed by dataplane switches is handled by the controller. Switches
are programmed by specifying a
forwarding table, where each row specifies a match (indicating a class of packets to which an entry applies) and a set of actions.
Switch actions are limited to forwarding packets (\ie send the packet a particular way), dropping packets (\ie choose to not forward packets) and
some limited mutation of the
packet header. On receiving a packet the switch uses the packet header to index the forwarding table and applies the specified set of actions.
Matches and actions depend only on the packet under consideration, \ie switches are stateless. Stateful processing in this model is performed at the
controllers, which receive any packets for which a switch does not have a forwarding entry.

\fixme{Some of this should move to section 4/5}
\fixme{Dataplanes stateful}
In reality, network dataplanes are more complicated: they commonly include a variety of stateful elements called middleboxes. Middleboxes
are dataplane elements implemented using general purpose computers (and programmed in C or other Turing complete languages) used to provide functionality not
expressible in switches. Middleboxes are used to improve network performance, security and add new features. Implementing such features in the dataplane is
essential for scalability: a datacenter network can have as many as 500k new flows every second~\cite{benson2010network} while a controller can only handle about
$5000-8000$ new  flows a second~\cite{}. Middleboxes are both common in the network (a recent study~\cite{sherry2012making} reports as many middleboxes as routers
in a network) and a common source of errors in the network~\cite{potharaju2013demystifying}: middlebox misconfigurations account for over $40\%$ of all major
incidents in networks. Directly using tools like HSA and NetKAT in such stateful networks is untenable: the rate of churn is too great to scale verification
in this manner. Further, naively extending these techniques to reason about state (instead of working on snapshots) renders their decision procedures undecidable.
Decision procedures for stateful networks must include techniques to handle unbounded sequences of packets and the state resulting from them and none of the
existing literature includes such techniques.

\fixme{Some of this should move to section 4/5}
\notepanda{Claim 8}
\fixme{Declarative languages don't work}
%% Start positive
De
These languages do not require programmers to specify a program's control flow, thus simplifying specification of stateless programs, enabling efficient
interpretation and simplifying runtime checking.

%% Where it breaks?

Therefore, both existing verification tools and network programming languages are insufficient once we move to stateful data paths. Modifying these to account
for state is non-trivial and existing tools hit a dead-end in the presence of stateful networks.

% \appendix
% \section*{Outline}
% \begin{outline}
% \1 Section 1: Introduction
%     \2 Verification done first for hardware, and now becoming more common for software
%         \3 But only recently have verification techniques been applied to networks
%         \3 These verification techniques have, for the first time, given us confidence that network configurations will produce ``correct" behavior.
%     \2 However, current verification work assumes statelessness (2 pages)
%         \3 Not representative of real networks
%         \3 Techniques cannot generalize to stateful networks
%         \3 Therefore, this is a dead-end
%     \2 Challenge we pose: how to verify stateful (i.e., real) networks? (3 pages)
%         \3 There are two main technical challenges
% \1 Section 2: Challenge \#1: How to model middleboxes? (to make feasible)
%     \2 Naive approach (of just verifying the code in each box) doesn't work
%         \3 Practical problem: large proprietary codebases, we'll never get them to verify
%         \3 Conceptual problem: what are you trying to verify?
%     \2 Operator's goals specified in terms of abstractions:
%         \3 Suspicious packets
%         \3 Packets belonging to a specific application
%         \3 Packets associated with a specific identify
%         \3 in-cache or not
%         \3 Etc.
%     \2 These abstractions usually not well defined
%         \3 And very implementation-specific (e.g., application identification, caching policy) and context-specific (identities)
%         \3 Much of the middlebox code is dedicated to implementing these abstractions
%     \2 To provide reasonable verification, we model middleboxes as being comprised of a Generic piece and an Oracle
%         \3 Generic: captures the basic logic in terms of these abstractions
%             \4 A reasonably small set of these, models can be developed without seeing the code
%         \3 Oracle: provides the mapping between packets/flows and a subset of these abstractions
%             \4 Specifies the set of abstractions implemented by the middlebox.
%             \4 We assume that the Oracle is correct in our verification.
%             \4 And consider it a separable problem to improve the behavior of the code devoted to these tasks
%             \4 cite Katerina (Doesn't quite verify these details, but close.)
%             \4 specify correctness condition, range of domain of oracle, etc.
%     \2 This approach has several advantages
%         \3 Generic models are both descriptive and enforceable
%         \3 Generic models can be easily developed
%         \3 Changes ecosystem
%             \4 Models come with mboxes, enable carrier checking
%         \3 But we need to specify invariants in order to know what level of abstractions....
%             \4 Reachability, isolation, (in terms of abstractions)....easy
%             \4 Correlations, ....easy
%             \4 Domain of Oracles
%             \4 Networks don't do much....don't worry about what we can't verify.
% \1 Section 3: Challenge \#2: How to analyze resulting network? (to make feasible)
%     \2 RONO (fill in details)
% \1 Section 4: Myths [Bad thing known by PL paper]
%     \2 Declarative languages are not a panacea for networking
%     \2 Reactive and proactive are not the only solutions.
%     \2 Centralization is not the answer.
%     \2 SDN is the only way to do verification \notepanda{I think we haave to be a bit careful}
%     \2 Learning switches and stateuful firewalls are ideally implemented in the controller
%     \2 Middleboxes are an abomination
% \1 Section 4: Things that are interesting from a PL and verification perspective
%     % \2 \fixme{Huh?} Cite DeMillo paper.... \fixme{Is this http://www.cs.umd.edu/~gasarch/BLOGPAPERS/social.pdf, doesn't it just say we are screwed}
%     \2 Formal semantics of a middlebox
%         \3 function from packet and history to a set of possible located packets.
%         \3 \fixme{Why Absract Middleboxes}
%     \2 Formal definition of verification problems
%         \3 Network isolation: common, important class of network invariants.
%         \3 Complexity of checking correctness.
%     \2 Compositional reasoning.
%         \2 Middleboxes as ADTs
%         \2 RONO as a first step to compositionality.
%         \2 Correctness preserving transformation: can we add or remove middleboxes without affecting correctness?
%     % \2 Think of MBoxes as ADTs (nice properties: modularity)
%     %     \3 Mboxes used to be seen as "hacks", but they should be seen as units of modularity
%     % \2 MBox models should use uninterpreted functions (so we don't need to model implementations, just generic functionality)
%     %     \3 This is how invariants will be expressed
%     %     \3 This is the natural way to build simple models
%     % \2 RONO is a first step towards compositional reasoning
%     %     \3 Our invariants are RONO-complete
%     %     \3 Some invariants aren't: how can we reason compositionally about them?

% \end{outline}
\end{document}
