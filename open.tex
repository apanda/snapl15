\section{Some Open Problems in Network Verification}
%\notepanda{We should consider changing the section title, Open Problems sounds very lofty.}
Finally, we present some open problems that we have encountered while looking at how to verify mutable dataplanes. This list is not 
exhaustive, but is rather an attempt to list the first set of hurdles that need to be crossed given this new network verification agenda.

\noindent\textbf{Decidability of Verification} When processing a packet, a middlebox might access potentially unbounded state. This prevents the use
of finite-state model checking, and other verification techniques are undecidable for general programs in this class. We are
currently working on a limited programming language that is rich enough to specify all existing middleboxes and to enable verification of some interesting network properties, including reachability properties. What other network properties can
be verified in a decidable manner remains an important open problem.

\noindent\textbf{Specification} While we have provided some tools
that allow us to specify and check reachability properties; extending this to other invariants, for example performance-based
invariants is challenging. How middleboxes and properties are specified also has a huge impact on verification time and
decidability. Therefore, it is crucial to pick specifications that are rich enough to permit operators to express interesting and
useful properties, yet narrow enough to permit automated reasoning.

\noindent\textbf{Correctness-Preserving Transformations} It might be possible to extend some of our results on compositional reasoning to show that the addition of certain types of middleboxes
can never affect some class of invariants. We know this is true for some middleboxes in reality, e.g., the addition of a stateless firewall
can never affect an isolation invariant (though it might invalidate some reachability invariants). Developing a theory for when this 
holds might be useful in developing techniques to help simplify network changes. 

\textbf{Verifying Parametric Topologies}
Some network topologies are parametric. For example, one can generate a fat-tree topology~\cite{al2008scalable} for a given datacenter
size. It is possible that we can leverage compositional verification techniques to verify properties independent of the parameter. This
would both speed-up verification and perhaps provide insights into the kinds of networks that are easily evolvable.