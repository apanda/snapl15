\section{Common Myths about Networks and SDN Verification}

We next highlight some common myths about SDN networks and contrast them with the reality of today's networks.

% \medskip{}

\subparagraph*{Myth \#1:} \emph{SDN networks only have controllers and OpenFlow routers, with all complicated (particularly mutable) packet processing done at the controller.} The early SDN literature~\cite{gude2008nox, monsanto2013composing} showed examples where anything expressible using OpenFlow rules was pushed down to routers, while anything more complex was implemented in the controller. Complicated functionality included both complex processing that could not be performed at routers and simple tasks that required mutability (\eg learning switches and stateful firewalls). However, doing this processing at the controller does not scale and, in reality, middleboxes are used to provide most of the processing functions not implementable on routers, and most routers provide some mutable behavior (e.g., learning switch).

\subparagraph*{Myth \#2:} \emph{Centralization, as provided by SDN, is what makes current network verification efforts possible.} Centralization is neither necessary nor sufficient for network verification. {\underline{Not necessary}:} Verification in a network with immutable datapaths only requires being able to access router forwarding state, and current commercial network verification efforts can do this in legacy networks by using commonly available commands to read this forwarding state. {\underline{Not sufficient}:} Regardless of SDN, current network verification efforts cannot verify networks that have middleboxes with mutable datapaths (which describes almost all real networks).


\subparagraph*{Myth \#3:} \emph{Middleboxes are an aberration that will be eliminated by the rise of SDN.} Quite the opposite is true. Not only are middleboxes here to stay, but SDN itself has been evolving to incorporate middleboxes~\cite{scottI2talk}. Furthermore, recently there has been an effort to move middleboxes from dedicated hardware (which is time-consuming to deploy) into virtual machines that can be deployed on quicker timescales, on existing hardware, and at lower cost. This effort is generally described as NFV (network function virtualization), and has gained significant traction commercially (comparable to or exceeding that of SDN), and recent efforts at defining a common configuration language, \eg Congress`\cite{congress}, treat middleboxes (virtualized or not) as first-class network citizens.


\subparagraph*{Myth \#4:} \emph{We should write all network code and configuration in declarative languages, because their use makes verification easy.} In general, reasoning about declarative languages is undecidable~\cite{Halevy}. It is true that verification is easy for declarative programs that do not use recursive rules (e.g., Congress~\cite{congress} or NLOG programs), even in the presence of mutable states. But then, verification is equally easy for imperative programs (e.g., Python, Java, or C programs) that honor certain restrictions, e.g., do not use loops. So, in the end, it is unclear that declarative languages can make a practical difference in verification. Some argue that declarative programs are easier to read and debug, once a programmer gets used to them. On the other hand, their readability becomes questionable in the presence of side-effects.

% \medskip

Once one discards these myths, it becomes clear that network-verification efforts must directly confront the presence of mutable datapaths. While the approach described here may not be optimal, it is currently the only one that confronts the reality of today's and tomorrow's networks. It is time to take the next step in network verification.
