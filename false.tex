\section{Common Myths about Networks and SDN Verification}

We next highlight some common myths about SDN networks and contrast them with the reality of today's networks.

\medskip{}

{\noindent \bf Myth \#1:} \emph{SDN networks only have controllers and OpenFlow routers, with all complicated (particularly mutable) packet processing done at the controller.} The early SDN literature~\cite{gude2008nox, monsanto2013composing} showed examples where anything expressible using OpenFlow rules was pushed down to routers, while anything more complex was implemented in the controller. Complicated functionality
included both complex processing that could not be performed at routers and simple tasks that required mutability (\eg learning switches and stateful firewalls). However, doing this processing at the controller does not scale and, in reality, middleboxes are used to provide most
of the processing functions not implementable on routers, and most routers provide some mutable behavior (e.g., learning switch).

{\noindent \bf Myth \#2:} \emph{Centralization, as provided by SDN, is what makes current network verification efforts possible.} Centralization is neither necessary nor sufficient for network verification. {\underline{Not necessary}:} Verification in a network with immutable datapaths only requires being able to access router forwarding state, and current commercial network verification efforts can do this in legacy networks by using commonly available commands to read this forwarding state. {\underline{Not sufficient}:} Regardless of SDN, current network verification efforts cannot verify networks that have middleboxes with mutable datapaths (which describes almost all real networks).

{\noindent \bf Myth \#3:} \emph{Middleboxes are an aberration that will be eliminated by the rise of SDN.} Quite the opposite is true. Not only are middleboxes here to stay --- as the NFV movement promoting their move to software has gotten significant commercial traction (comparable to or exceeding that of SDN) and configuration efforts like Congress \cite{congress} are treating them as first-class network citizens --- but SDN itself has been evolving to incorporate middleboxes~\cite{scottI2talk}.

{\noindent \bf Myth \#4:} \emph{We should write all network code and configuration in declarative languages, because their use makes verification easy.} In general, reasoning about declarative languages is undecidable~\cite{Halevy}. It is true that verification is easy for declarative programs that do not use recursive rules (e.g., Congress~\cite{congress} or NLOG programs), even in the presence of mutable states. But then, verification is equally easy for imperative programs (e.g., Python, Java, or C programs) that honor certain restrictions, e.g., do not use loops. So, in the end, it is unclear that declarative languages can make a practical difference in verification. Some argue that declarative programs are easier to read and debug, once a programmer gets used to them. On the other hand, their readability becomes questionable in the presence of side-effects.

\eat{
{\noindent \bf Myth \#4:} \emph{We should write all network code and configuration in declarative languages, because their use makes verification decidable.} The literature has assumed that SDN controllers are written using Turing complete languages (C, Python, etc.) and are used to control an immutable network with limited capabilities.  Using declarative languages to express controllers for these networks might have some benefits, however
reasoning about declarative languages is undecidable (\eg see the work from \cite{Halevy}). However, using declarative programs for specifying network configuration as done in Congress~\cite{congress} and NLOG makes verification easier: these specifications only use non-recursive rules, however similarly restricted loop-free C code is equally simple to verify~\cite{cbmc}. Recursion is often essential to implement more general network functionality, and cannot be expressed
using these limited languages.}

\medskip

Once one discards these myths, it becomes clear that network verification efforts must directly confront the presence of mutable datapaths. While the approach described here may not be optimal, it is currently the only one that confronts the reality of today's and tomorrow's networks. It is time to take the next step in network verification.


\eat{
\begin{outline}
\1 SDN networks only have controllers and OpenFlow routers, with all complicated (particularly mutable [make point about complicated and mutable two different things]) processing done at the controller (e.g., learning switches, firewalls).
    \2 Centralizing it is unscalable, and Middleboxes keep some (all?) mutable processing in the network; an approach already in widespread use
\1 SDN's centralization makes verification possible
    \2 Centralization is neither necessary nor sufficient for verification
    \2 Commercial verification efforts deal with legacy (just read routing state)
    \2 Centralization not sufficient for middleboxes
\1 Middleboxes are an abomination that SDN will eliminate
    \2 Get real: they are widely deployed, essential for security and performance. NFV is outstripping SDN in terms of deployment.
    \2 Instead, SDN will (and already has) evolve to incorporate middleboxes (see Scott's talk)
\1 Write all network code and configuration in declarative languages, since declarative $\implies$ decidable checking of properties
    \2 Not a good way to write middleboxes: Declarative not as useful in the presence of mutable state and side-effects~\cite{Tjgreen}.
        \3 State mutation is imperative in nature and control flow of a program specific ordering and such
        \3 Hard to call out to imperative code, etc. none of the existing work in building middleboxes is easily adaptable.
    \2 Perhaps reasonable for specifying network configuration~\cite{congress} but does not help with verification with mutable datapaths
        \3 Dataplane is Turing complete (as opposed to OpenFlow), controller complexity has nothing to do with undecidability
        \3 Use it if it works for you, not a magic bullet.
\end{outline}
}
% \begin{description}
% \item [Myth] The only options for packet handling are reactive and proactive.
% \item [Reality]  Reactive is usually not an option and proactive may not be powerful enough. Middleboxes compliments the two and is/will be widespread.
% \end{description}

% \begin{description}
% \item [Myth] Centralized solutions can scale well in networks by periodically installing rules.
% \item [Reality] Middleboxes enable scalability while allowing complex distributed functionality using mutable states.
% \end{description}

% \begin{description}
% \item [Myth] SDN opens up the possibility for network verification.
% \item [Reality] The OpenFlow drastically simplifies network verification but it is not realistic and thus any verification effort which can \textbf{only} handles
% OpenFlow has limited applicability.
% \end{description}


% \begin{description}
% \item [Myth] Learning switches and statful firewalls need to be implemented in the controller.
% \item [Reality] Learning switches are some of the simplest and most common dataplane elements and are not normally implemented using a controller. A stateful
% firewall is a very common middlebox.
% \end{description}

% \begin{description}
% \item [Myth] The use of declarative languages renders network verification problems decidable.
% \item [Reality] . Declarative programming languages have been used to effectively express network policies in languages like Nlog~\cite{Nlog},
% Flowlog~\cite{flowlog}, NDLog~\cite{}, NetCore~\cite{monsanto2012compiler} and have also been recently used to describe network configuration~\cite{Congress}.
% However, these languages loose their charm in the presence of side-effects and mutable state~\cite{Tjgreen}.
% State mutation is imperative by nature and the control flow of a program is used to specify how these updates are processed. Specifying state updates
% using a declarative language quickly becomes counterproductive as a result.
% Furthermore, even simple updates result in a Turing complete language~\cite{Janos}, thereby preventing the compiler from handling
% \textbf{all} programs. Further, calling imperative code from declarative programming is challenging and further prevents optimization and runtime checking.
% Finally, we note that these languages lack mechanisms for modularity such as procedures, modules, packages, and classes which are
% heavily used in building middleboxes and complex network code.
% \end{description}

% \begin{description}
% \item [Myth] Middleboxes are abomination
% \item [Reality] Middleboxes are necessity. Half of the network;
% Solving real and pressing problems;
% Needs that are not likely to go away;
% Enable local functionality enhancements~\cite{Riverved}
% \end{description}
