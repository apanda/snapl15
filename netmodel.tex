\section{Network-Wide Verification}
\label{sec:modelnet}

Modeling individual middleboxes is only the first step; our ultimate goal is to verify network-wide invariants in a network containing middleboxes and routers.
There are numerous technical challenges (such as how to deal with loops), but in this short position paper we focus on the most challenging one: how can we feasibly perform verification in very large networks (containing on the order of thousands of middleboxes and routers)? To see why this is hard, consider the case of an isolation invariant (packets from host A cannot reach host B) in a network with many middleboxes. Since middlebox behavior depends not just on their configuration but on the packet history they've seen (since their datapath is mutable), verifying that this invariant holds, even if we ignore the possibility of packet loops,
involves checking that there is no sequence of packets --- involving packets sent from anywhere in the network at any time --- which includes a packet from host A reaching host B.


Without additional assumptions, this is not feasible as the forwarding of packets from host A to host B can arbitrarily depend on other hosts (e.g., on whether some other host C previously sent packets to another host D). However, we note that the most common classes of middleboxes have an important property: the handling of a packet from host A to host B depends only on the sequence of packets (seen by the middlebox) between hosts A and B. That is, many mutable datapaths exhibit a useful kind of locality.
For instance, in IP
firewalls, the middlebox tracks established connections, and allows packets to pass if they are either explicitly permitted by
policy or belong to an established connection. A connection between host A and B can only be established by host A and B. We therefore do not need to consider the
actions of any other hosts in the network. We call middleboxes whose behavior for a pair of hosts depends only on the traffic sent between these hosts ``RONO (Rest-Of-Network Oblivious) middleboxes''. See Section~\ref{sec:formal} for a formal definition of RONO. RONO is not a rare property; in fact, most middleboxes (including firewalls, WAN optimizers, load balancers, and others) are RONO.  Moreover, one can verify whether a middlebox is RONO by statically analyzing its abstract model.


Note that in many practical cases, 
the composition of RONO middleboxes is also RONO.  As a result, in a network containing only RONO
middleboxes we can verify reachability properties on a small subset of the network (the path between two hosts) and these properties would equivalently hold in
the context of the wider network. We have leveraged this fact to verify correctness of a network containing 30,000 middleboxes in under two minutes.


\eat{

RONO middleboxes are not rare in-practice and several non-RONO middleboxes (such as content caches) are extended to make them RONO. Common implementations of
content caches accept ACL rules to limit where content can be accessed from and where correctly configured these act as RONO equivalent middleboxes. One can
statically verify if a model is RONO.

While RONO is one property that allows faster verification of properties in such networks, we believe other properties such as compositionality or commutativity
might also apply to several middleboxes and might enable faster verification. In the next section we talk about some properties that we believe (but have not
proven) hold for stateful dataplane element and help with network verification.

We do not have a general solution for this problem, but have an approach that can scalably verify simple reachability and isolation invariants under certain restrictions.
}
\eat{Now that we have described our models of individual middleboxes, we turn to modelling the network itself. This is harder than reasoning about stateless
networks for a few reasons
\begin{enumerate}
\item When dealing with mutable state, we must account for more than just a single packet. We thus must reason about all paths through the network
      and their effect on state, this might affect both scalability and decidability when verifying properties.
\item Temporal effects become important, in particular the order of packets processed by a middlebox affects its state.
\item Even when dealing with a single packet, network loops might require that we reason about unbounded sequences of data path elements. What makes this worse
      than the stateless case is that every loop through the network might affect the state in the data path. Concretely this means we cannot deal with loops using
      the Kleene star (as is suggested in NetKAT~\cite{anderson2014netkat}), since each iteration cannot be modeled as applying the same policy repeatedly.
      Further, stateful loops imply that the datapath might be Turing complete and verifying any property might easily become undecidable.
\item Even loop free networks can be huge, involving several 10s of thousands of switches, middleboxes, etc. The size is a challenge for scalability.
\end{enumerate}
In this section we present a few techniques we have used to deal with these challenges.

First, we deal with simple forwarding loops, \ie forwarding loops that are caused by static configuration, for instance loops that are obvious from
examining the forwarding tables of switches, by leveraging existing work. To do this, given a network topology with middleboxes, we create an equivalent
topology where the middlebox is replaced by a switch with the same forwarding behavior as the middlebox: \ie the equivalent switch forwards a given packet
the same way the middlebox would if it had output the same packet \notepanda{Fix wording}. We use Veriflow (HSA or other tools could be used as easily) on this
equivalent topology to verify that they are loop-free. Our current tool can only operate on loop-free topologies so we report a verification when this is not
the case. Analyzing static configuration is also more scalable, so we also use the output of these tools (in loop-free toplogies) to eliminate switches: we generate
network transfer functions between all pairs of middleboxes and hosts in a network. We
then derive a network topology where all switches are replaced by a single switch that implements this transfer function. This topology is equivalent to the
original network topology (as has been shown in HSA, VeriFlow, and other works).


Even in this simplified, loop free topology, verifying a property might require analyzing $O(2^n)$ for a network with $n$ middleboxes.
While we do not have a general solution for this problem, we have a specific solution that allows us to verify an important class of properties in networks which
only contain a limited type of middleboxes.
Our solution is applicable to properties which reason about transfers between two hosts (or classes of hosts). Examples of such invariants include:
\begin{itemize}
\item Can host A send packets to host B?
\item Can a host outside the enterprise connect to the mail server?
\item Can host A access data that originated at host B?
\end{itemize}
Note, for stateless data planes, these properties can be checked by traversing all paths between the two hosts (or classes of hosts) and verifying that they hold
(or finding a counterexample). This is not as simple once state comes into play, any communication that goes through the middlebox might affect its behavior and
hence the entire network must be analyzed.
}
